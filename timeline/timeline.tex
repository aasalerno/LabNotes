\documentclass[11 pt]{article}

\usepackage{titlesec}
\usepackage[dvipdfmx]{graphicx}
\usepackage{bmpsize}
\usepackage{amsmath,mathrsfs}
\usepackage{amssymb}
\usepackage{enumerate}
\usepackage{grffile}
\usepackage{hyperref}
\hypersetup{
    colorlinks,
    citecolor=black,
    filecolor=black,
    linkcolor=black,
    urlcolor=black
}
\usepackage[margin=1in]{geometry}
\usepackage[font={scriptsize}]{caption}


\newcommand{\me}{\mathrm{e}}
\newcommand{\ep}{\varepsilon}
\newcommand{\sectionbreak}{\clearpage\newpage}
\newcommand{\mitem}{\item[--]}
\newcommand{\bo}{\noindent\textbf}
\newcommand{\etal}{\emph{et. al.}}
\newcommand{\matlab}{\textsc{Matlab }}
\let\oldsection\section
\renewcommand\section{\clearpage\newpage\oldsection}

\usepackage[nodisplayskipstretch]{setspace}
\usepackage{fancyhdr}
\pagestyle{fancy}

\fancyhead[R]{Salerno}
\fancyhead[C]{Lab Notes for Compressed Sensing Coding}
\fancyhead[L]{MICe}

\begin{document}

{\large{Timeline}}

      
\begin{enumerate}

      \item \bo{Finish with simulations}\\
	    \emph{Due: 04.02.15}
	    \begin{itemize}
      
		  \item As of 04.01.15 I think I'm done with these... The best method has been figured out and it's just doing it via the circle
      
	    \end{itemize}
      
      
      \item \bo{Figure out parameterization choices for CS}\\
	    \emph{Due: 04.15.15}
	    \begin{itemize}
		  \item Understand what each parameter does fully
		  \item Figure out why I'm getting poor results on the CS
		  \item Likely a good idea to go through the math again -- figure out what is happening in the CS code
		  \item Spend a day or two porting?
		  \item Reading MINC files in would be easier
		  \item More understanding around the lab on how it works 
		  \item However, I'm not as proficient in python...
		  \item \bo{THIS SHOULD BE DONE IN ONE WEEK (giving two to be safe)}
	    \end{itemize}
      
      
      \item \bo{Massage in 3D reconstructions}\\
	    \emph{Due: 04.29.15}
	    \begin{itemize}
		  \item Analyze the 3D wavelet work that can be done in MATLAB
		  \item Need to understand how the p2DFT and XFM classes work in Lustig's code in order to adapt them to work in 3D
		  \item Run reconstructions on full data that has been undersampled
		  \item Keep in mind that the undersampling doesn't occur in the readout direction
	    \end{itemize}

      
      \item \bo{Look at different methods of adding in the extra term in the reconstruction}\\
	    \emph{Due: 04.29.15} (possibly could be between \#2 and \#3)
	    \begin{itemize}
		  \item Is there anything that can be done to make the reconstruction have this extra term easily?
		  \item A good form may be $\lambda_3 ||m_j - m_k||_2 (\vec{d_j} \cdot \vec{d_k})^2$
	    \end{itemize}

\end{enumerate}

\end{document}